\documentclass[10pt,a4paper]{fancyhandout}
	\usepackage[T2A]{fontenc}
	\usepackage[english,ukrainian]{babel}
	\usepackage{indentfirst}
	\usepackage{graphicx}
	\usepackage{qrcode}
	\usepackage{hyperref}
	\usepackage{enumitem}
	\usepackage{footnote}
	\usepackage{listings}
	\makesavenoteenv{tabular}
	\usepackage[
    	type={CC},
    	modifier={by-nc},
    	version={3.0},
	]{doclicense}
	\graphicspath{{pictures/}}
	\title{\textbf{Короткий посібник з курсу Linux}}
	\author[О. Фоменко]{Олексій Фоменко}
	\date{}
	\addtolength{\topmargin}{-.5cm}
	\addtolength{\textheight}{.5cm}
\begin{document}
\maketitle
\section{Вступ}
Привіт, друзі! Вітаю вас. У цьому курсі на вас чекає багато інформації від мене та моїх колег, тому для того, щоб вам було зручніше орієнтуватись, я зробив цей посібник, щоб ви мали можливість звертатись до нього в разі потреби як під час навчання, так і після нього. Ця брошура охоплює лише одну тему з курсу, а саме -- <<операційна система Linux>>. В додатку буде наведено домашнє завдання, що я вам надаватиму під час курсу та довідник з основних команд, що використовуються у буденній роботі як системного адміністратора, так і девопса. \\
Щиро бажаю вам успіхів у навчанні!

\subsection{Маленька порада}
Для того, щоб якісно засвоїти теми, що викладатимуться під час курсу, дуже раджу вам створити та використовувати тестове оточення (в прикладах буде наведено використання VirtualBox, але ви можете використовувати будь-який гіпервізор на свій смак) навіть якщо ви використовуєте Linux щодня на своєму комп'ютері.

\section{Коротко про Linux}
Linux почався як аматорський проєкт однієї людини на ім'я Линус Торвальдс у далекому 1991 році і за цей час набув неабияку популярність у суспільстві, зокрема у великих бізнесах. Станом на сьогодні Linux опанував веб-сервери (понад 37\% станом на травень 2022 року); суперкомп'ютери (100\% суперкомп'ютерів із списку Top 500); одноплатні комп'ютери, що використовуються не лише для дослідницьких або розважальних цілей, але й для прикладних задач; промислове обладнання; ігрові консолі; системи навігації; смартфони (так, Android теж походить від Linux); маршрутизатори та купу іншого обладнання (навіть пральні машини та холодильники), де показав себе як дуже зручна, гнучка та надійна операційна система.
\footnote{\textit{Варто зазначити, що правильніше операційну систему називати GNU/Linux, бо Linux -- це лише ядро, що дозволяє обробляти системні виклики, а для взаємодії з людиною найчастіше використовується певна кількість програм, що було створено проєктом GNU (в рамках проєкту було створено майже 400 пакунків, що розповсюджуються під вільною ліцензією GNU GPL), тому засновник цього проєкту Річард Столман наполягає на тому, що будь-який дистрибутив Linux обов'язково треба називати GNU/Linux, але це питання більш етичне ніж технічне). Крім програм проєкту GNU Лінус Торвальдс використав для розповсюдження Linux ліцензію, що було розроблено тим проєктом -- GNU GPL (General Public License) v.2, що стало запорукою популярності його спроби написати власну операційну систему, бо кожен бажаючий міг (і, звісно, може досі) вільно отримувати, користуватись, модифікувати та розповсюджувати результати спочатку особисто його праці, а згодом і сотень тисяч програмістів у всьому світі.}}
\goodbreak

\section{Інструменти для керування репозиторіями та окремими пакунками}
На відміну від більш звичних для багатьох операційних систем Microsoft Windows та Apple MacOS X, програмне забезпечення в різних дистрибутивах Linux розповсюджується централізовано, з використанням репозиторієв. Репозиторій (або скорочено <<репо>> -- це сховище пакунків програм.
\footnote{\textit{В українській мові досі немає сталого терміну, що відповідає англійському package, то ж в українських локалізаціях різних дистрибутивів використовується або <<пакет>>, або <<пакунок>>. Особисто я вважаю більш правильним слово <<пакунок>> з двох причин: щоб не плутати, скажімо, з мережевими пакетами і, головне, package є саме запакованим в архів програмним забезпеченням.}}
Різні дистрибутиви використовують різні формати пакунків та, відповідно, різні інструменти для роботи з ними. Станом на сьогодні все різноманіття дистрибутивів Linux можна умовно поділити на дві великі групи:
\begin{itemize}
	\item Ті, що використовують формат пакунків \textit{deb}: Debian, Ubuntu, Linux Mint та інші, що засновані на Debian;
	\item ті, що використовують формат пакунків \textit{rpm}: Red Hat Enterprise Linux (RHEL), CentOS, Alma Linux, Rocky Linux та інші, що засновані на RHEL.
\end{itemize}
Звісно, є інші дистрибутиви (наприклад, Arch Linux або NixOS), що користуються своїми форматами пакунків, але так історично склалось, що в комерційних або дослідницьких цілях найбільш популярними є ці два варіанти. \\
Трохи прикладів з керування пакунками в різних дистрибутивах: \\
\begin{tabular}{|l|l|l|}
	\hline
	& \texttt{rpm} & \texttt{deb} \\\hline
	Показати інформацію про репозиторії & 
		\begin{tabular}{@{}l@{}}\texttt{dnf repolist}\\\texttt{dnf repolist all} \footnote{\textit{{\texttt{dnf repolist all} показує зокрема репозиторії, що не є активними в системі}}}\end{tabular} & \texttt{} \\\hline
	\begin{tabular}{@{}l@{}}Показати всі пакунки, \\що встановлені в системі\end{tabular} & \texttt{dnf list installed} & \texttt{apt list --installed}\\\hline
	Оновити інформацію про пакунки & \texttt{dnf check-update} & \texttt{apt update} \\\hline
	Повністю оновити систему & \texttt{dnf update} & \texttt{apt upgrade} \\\hline
	Пошук пакунка & \texttt{dnf search ім'я\_пакунка} & \texttt{apt search ім'я\_пакунка} \\\hline
	\begin{tabular}{@{}l@{}}Встановлення пакунка, \\ що знаходиться локально\end{tabular} \footnote{\textit{{Для встановлення пакунка, що знаходиться локально, треба вказувати повний (абсолютний) шлях до пакунка}}} &  \texttt{dnf localinstall ім'я\_пакунка} & \texttt{apt install ім'я\_пакунка} \\\hline
	Вилучення пакунка, що встановлено & \texttt{dnf remove ім'я\_пакунка} & \texttt{apt remove ім'я\_пакунка} \\\hline
	\begin{tabular}{@{}l@{}}Вилучити пакунки,\\що не використовуються системою\end{tabular} & \begin{tabular}{@{}l@{}}\texttt{dnf clean all} \\\texttt{dnf autoremove}\end{tabular} & \texttt{apt autoremove} \\\hline
\end{tabular}
\goodbreak

\section{Текстовий інтерфейс}
Основним інструментом взаємодії із сервером з боку системного адміністратора є консоль або термінал.
\footnote{\textit{В переважній більшості випадків це буде не консоль як така, а емулятор консолі або емулятор терміналу, але відмінності тут в першу чергу термінологічні.}}
Цей інструмент надає доступ до командної оболонки операційної системи; емулятор терміналу -- це одна з небагатьох програм, що встановлюється під час інсталяції операційної системи по замовчанню. \\
На відміну від засобів адміністрування з графічним інтерфейсом для роботи в консолі (а особливо для ефективної роботи) треба знати основні команди, що використовуються щодня. \\
В більшості випадків команди мають таку структуру: \\
\texttt{Команда Опції Параметри} \\
де \texttt{Команда} -- це, власне, правило, що ви хочете застосувати; \\
\texttt{Опції} -- модифікатори команди, зазвичай вони задаються через дефіс або два дефіси; \\
\texttt{Параметри} -- необхідна інформація для команди. \\
Зазвичай опції можна комбінувати разом, наприклад, \texttt{-abcdef} буде працювати так само, як \texttt{-a -b -c -d -e -f}. \\
Повну інформацію про команду можна дізнатись з допомогою довідника \texttt{man}: \\
\texttt{man <command>}\\
\textit{Майте на увазі, що всі команди в Linux розрізняють великі літери від маленьких, щоб не отримати неочікуваний результат!}
\goodbreak

\subsection{\texttt{sudo}}
\texttt{sudo} є скороченням від superuser do, це одна з команд, якою користуються найчастіше. Вона дозволяє виконати будь-яку задачу, що потребує прав адміністратора системи (в UNIX-подібних системах він називається root) або іншого користувача. \\
Під час використання цієї команди з метою безпеки система вимагає від користувача його пароль. По замовчанню час дії сеансу \texttt{sudo} триває п'ятнадцять хвилин, після чого у вас знову спитають пароль коли ви захочете виконати \texttt{sudo}. \\
Типовий варіант використання \texttt{sudo}: \\
\texttt{sudo <команда>} \\
Основні опції, що можуть застосовуватись до команди: \\
\texttt{-u <user>} або \texttt{~---user=<user>} -- виконати команду з правами вказаного користувача, без цієї опції користувач є адміністратором (root); \\
\texttt{-g <group>} або \texttt{~---group=<group>} -- те саме, але для групи.
\goodbreak

\subsection{\texttt{pwd}}
Команда виводить абсолютний (тобто починаючи з кореневої) шлях директорії, де ви наразі знаходитесь.
\texttt{\$ pwd} \\
\texttt{/home/user}
\goodbreak

\subsection{\texttt{cd}}
Для навігації файловою системою Linux використовується команда \texttt{cd} (change directory). В залежності від вашої чинної директорії вона потребує або абсолютний, або відносний (відносно директорії, де ви знаходитесь просто зараз) шлях. \\
Виконання команди без опцій перенесе вас у вашу домашню директорію так само як \texttt{cd $\sim$\slash}\\
\texttt{cd /var/log} -- перейти в директорію /var/log; \\
\texttt{cd Downloads} -- перейти в директорію Downloads, що знаходиться у вашій робочій директорії (якщо її там немає, то ви отримаєте сповіщення про помилку); \\
\texttt{cd $\sim$<username>} -- перейти в домашню директорію користувача <username>; \\
\texttt{cd ..} -- перейти в директорію на рівень вище; \\
\texttt{cd -} -- перейти в попередню директорію.
\goodbreak

\subsection{\texttt{ls}}
Команда видає список файлів і директорій в системі; запуск команди без опцій виводить вміст робочої директорії. Для того, щоб подивитись вміст іншої директорії, треба додати абсолютний або відносний шлях, наприклад: \\
\texttt{ls /home/user/Downloads} \\
\texttt{ls -R} -- виводить вміст усіх піддиректорій; \\
\texttt{ls -a} -- додатково показує файли і директорії, що є прихованими (hidden);
\footnote{\textit{В Linux прихованим є будь-який файл, ім'я якого починається з крапки. }} \\
\texttt{ls -l} -- повний формат виводу; \\
\texttt{ls -lh} -- повна інформація, але розмір файлів указується в <<людському>> формати (мегабайти, гігабайти та терабайти).
\goodbreak

\subsection{\texttt{cat}}
Команда, що виводить по замовчанню на стандартний вивід (переважно на екран) вміст файлів: \\
\texttt{cat <filename>} -- вивести файл на екран; \\
\texttt{cat <filename1> <filename2> > <filename3>} -- об'єднати файли <filename1> і <filename2> і зберегти результат у файлі <filename3>; \\
\texttt{tac <filename>} -- показати вміст файлу в зворотньому порядку.
\goodbreak

\subsection{\texttt{cp}}
Команда для копіювання файлів, директорій та їхнього вмісту. \\
\texttt{cp <filename> /home/user/Downloads} -- скопіювати файл <filename> в директорію /home/user/Downloads; \\
\texttt{cp <filename1> <filename2> <filename3> /home/user/Downloads} -- скопіювати файли <filename1>, <filename2> та <filename3> в директорію /home/user/Downloads; \\
\texttt{cp <filename1> <filename2>} -- скопіювати файл <filename1> і зберегти його з іменем <filename2>; \\
\texttt{cp -R /home/user/Downloads /home/user/Backup} -- скопіювати директорію /home/user/Downloads з усім вмістом в директорію /home/user/Backup.
\goodbreak

\subsection{\texttt{mv}}
Команда для переміщення файлів та зміни їхнього імені. \\
\texttt{mv <filename> /home/user/Downloads} -- перемістити файл <filename> в директорію /home/user/Downloads; \\
\texttt{mv /home/user/Downloads/<filename1> /home/user/Downloads/<filename2>} -- перейменувати файл <filename1>, що знаходиться в директорії /home/user/Downloads у <filename2> в тієї же директорії. 
\goodbreak

\subsection{\texttt{mkdir}}
Команда для створення директорій. \\
\texttt{mkdir /home/user/Downloads} -- створити директорію Downloads, що розташована в /home/user; \\
\texttt{mkdir -p /home/user/Downloads/Linux/Fedora} -- послідовно створити директорії (якщо вони не існують) Downloads, Linux і Fedora в /home/user/; \\
\texttt{mkdir -m700 /etc/secrets} -- створити директорію secrets в /etc і задати дозвіл 700 для неї.
\footnote{\textit{Про дозволи і що означають ці цифри буде написано трохи пізніше. }}
\goodbreak

\subsection{\texttt{rmdir}}
Команда для видалення директорій.
Слід пам'ятати, що \texttt{rmdir} видаляє лише пусті директорії, якщо всередині буде будь-який файл або інша директорія, то ви отримаєте помилку \texttt{Directory is not empty}. \\
\texttt{rmdir /home/user/Downloads} -- вилучити пусту директорію Downloads в /home/user;\\
\texttt{rmdir -p /home/user/Downloads/Linux} -- послідовно вилучити пусті директорії /home/user/Downloads/Linux та /home/user/Downloads.  
\goodbreak

\subsection{\texttt{rm}}
Команда для видалення файлів. Будьте обережні, ця команда може бути занадто потужною, завжди перевіряйте шляхи, що вказуєте як параметр. \\
\texttt{rm <filename>} -- вилучає файл. У багатьох дистрибутивах для цього команда спитає підтвердження, чи ви дійсно хочете вилучити цей файл. Така поведінка не завжди є зручною, тому можна її змінити з допомогою опції \texttt{-f} (force): \\
\texttt{rm -f <filename>} -- вилучення файлу без підтвердження; \\
\texttt{rm <filename1> <filename2> <filename3>} -- вилучення декількох файлів одночасно; \\
\texttt{rm -r /home/user/Downloads} -- рекурсивне вилучення директорії Downloads разом із усім її вмістом; \\
\texttt{rm -rf /home/user/Downloads} -- те саме, але без підтвердження дії. \\
\textit{Ще раз нагадаю, що \texttt{rm -rf} є дуже потужною командою, що вилучає файли і директорії без підтверджень, то ж перевіряйте декілька разів шлях, до якого ви її застосовуєте!}
\goodbreak

\subsection{\texttt{touch}}
Команда створює пустий файл або (якщо застосувати до наявного) змінює час його створення. \\
\texttt{touch /home/user/Downloads/filename}
\goodbreak

\subsection{\texttt{locate}}
Пошук текстових файлів, що містять патерн, який вказується як параметр; для того, щоб вказати більше за одне слово, використовуйте символ \texttt{*}: \\
\texttt{locate слово1*Слово2} -- шукатиме файл, в якому є <<слово1>>, що написано з маленької літери, і <<Слово2>> -- з великої; \\
\texttt{locate -i слово} -- пошук файлів, що містять <<слово>>, ігноруючи регістр літер, якими воно написано. 
\goodbreak

\subsection{\texttt{find}}
Команда для пошуку файлів всередині вказаної директорії; дозволяє робити складні запити з допомогою регулярних виразів, але базові приклади застосування є досить простими: \\
\texttt{find -name filename} -- пошук файлів і директорій з назвою <filename> в директорії, де ви знаходитесь; \\
\texttt{find / -type d -name directoryname} -- пошук директорій з назвою <directoryname> на всіх розділах, що є в системі.
\goodbreak

\subsection{\texttt{grep}}
Команда, що шукає в файлі рядки, які містять патерн, вказаний як опція. Це ще одна команда, функціонал якої може збагачуватись за допомогою регулярних виразів. \\
\texttt{grep слово filename} -- простий пошук рядків, у яких є <<слово>>; \\
\texttt{grep -i слово filename} -- те саме, але ігноруючи регістр.
\goodbreak

\subsection{\texttt{df}}
Команда показує (у відсотках та кілобайтах) поточне використання дисків у системі: \\
\texttt{df -h} -- виводить розмір в зручному форматі (мегабайти, гігабайти, терабайти); \\
\texttt{df -T} -- додає у вивід стовпчик, в якому вказано тип файлової системи.
\goodbreak

\subsection{\texttt{du}}
Команда для перевірки розміру директорії: \\
\texttt{du /home/user/Downloads} виводить загальний розмір директорії Downloads і всіх директорій, що є вкладеними; \\
\texttt{-h} змінює розмір з кілобайтів на мегабайти, гігабайти та терабайти; \\
\texttt{-d 2} задає максимальну вкладеність піддиректорій (у даному випадку максимум дві); \\
\texttt{-s} виводить розмір лише директорії, що вказано.
\goodbreak

\subsection{\texttt{head}}
По замовчанню виводить перші десять рядків на екран. \\
\texttt{head -n 5 <filename>} або \\ \texttt{head ~---lines 5 <filename>} виводить перші п'ять рядків файлу <filename>; \\
\texttt{head -c 40 <filename>} або \\ \texttt{head ~---bytes 40 <filename> виводить перші сорок байт файлу <filename>.}
\goodbreak

\subsection{\texttt{tail}}
Працює так само, як \texttt{head}, але виводить не перші, а останні рядки файлу. Опції використовуються так само, але є один нюанс: опція \texttt{-f} не лише виводить останні рядки файлу, а й продовжує це робити в реальному часі допоки не буде зупинено. Це дуже зручно коли треба відстежувати поведінку якогось сервісу виводячи на екран (або у файл) системний лог.
\goodbreak

\subsection{\texttt{diff}}
Команда \texttt{diff} порівнює вміст файлів рядок за рядком. \\
\texttt{diff file1 file2} \\
\texttt{-i} порівнює файли ігноруючи регістр, у якому написані слова; \\
\texttt{-c} показує різницю в контекстній формі.
\goodbreak

\subsection{\texttt{tar}}
Утиліта, що архівує декілька файлів (або директорій) в один; додатково може бути застосовано стискання задля економії місця на диску. \\
\texttt{tar -cvf archive.tar /home/user/Downloads} створює файл \texttt{archive.tar} з вмістом директорії \\\texttt{/home/user/Downloads}; \\
\texttt{-c} -- створити архів; \\
\texttt{-v} -- вказувати файли, що додаються в архів; \\
\texttt{-x} -- розархівувати файл; \\
\texttt{-u} -- додати в архів файли, новіші за ті, що вже є в архіві; \\
\texttt{-t} -- показати вміст архіву.
\goodbreak

\subsection{\texttt{chmod}}
Загальна команда, що модифікує дозволи (читання, запис, виконання) для файлів або директорій; детальніше про використання команди читайте в розділі \ref{Безпека та дозволи}.
\goodbreak

\subsection{\texttt{chown}}
Команда, що дозволяє змінити власника файлу, директорії або символічного посилання на вказаного користувача; детальніше про використання команди читайте в розділі \ref{Безпека та дозволи}.
\goodbreak

\subsection{\texttt{kill}}
Команда для примусового завершення роботи програми (зазвичай, що не відповідає) вручну; вона надсилає специфічний сигнал застосунку з вимогою зупинити процес (або процеси). \\
Для використання утиліти треба вказати номер процесу (process ID або просто PID), щоб його визнати скористуйтесь командою \texttt{ps ux} \\
Існує 64 сигнали, але найчастіше використовуються \texttt{SIGTERM} (звичайне завершення роботи процесу) та \texttt{SIGKILL} (припинення роботи програми та всіх похідних від неї процесів): \\
\texttt{kill <сигнал> PID} \\
\texttt{kill PID} -- надсилання сигналу \texttt{SIGTERM}; \\
\texttt{kill -9 PID} -- надсилання сигналу \texttt{SIGKILL} \\
Пам'ятайте, що \texttt{SIGKILL} є дійсно аварійною зупинкою роботи програми, то ж жодні дані, що залежать від неї, не буде збережено, то ж не варто використовувати цей сигнал <<по замовчанню>>.
\goodbreak

\subsection{\texttt{ping}}
Команда э однією з команд, що використовується в першу чергу для перевірки мережевого з'єднання або перевірки доступності віддаленого комп'ютера. Звичайний формат використання: \\
\texttt{ping <host>} де \texttt{host} є іменем комп'ютера або його IP адресою.
\goodbreak

\subsection{\texttt{wget}}
Утилита, що дозволяє завантажувати файли використовуючи протоколи HTTP, HTTPS або FTP. Зокрема підтримується рекурсивне завантаження, таким чином можна скопіювати локально веб-сайт разом із структурою директорій, що його створюють. \\
\texttt{wget https://www.python.org/ftp/python/3.12/Python-3.12.0.tgz} -- завантажити архив Python-3.12.0.tgz з офіційного сайту проєкту Python і зберегти його в поточну діректорію
\goodbreak

\subsection{\texttt{uname}}
Вивід короткої інформації про систему: \\
\texttt{-n} -- мережеве ім'я комп'ютера; \\
\texttt{-r} -- номер релізу ядра операційної системи; \\
\texttt{-a} -- повна (насправді ні) інформація про ядро, архітектуру комп'ютера та мережеве ім'я.
\goodbreak

\subsection{\texttt{top}}
Команда для відображення всіх процесів системи в реальному часі, інформації про час роботи комп'ютера, використання процесорів та пам'яти; команда дозволяє визначити та зупинити процеси. 
\goodbreak

\subsection{\texttt{history}}
Виводить останні 500 команд, що було виконано. Зверніть увагу, кожний рядок має свій номер, то ж якщо треба виконати якусь команду з історії, достатньо написати \\
\texttt{! <номер команди>} для її повторного запуску. \\
\texttt{history -c} -- вилучення всієї історії; \\
\texttt{history -a} -- примусове додавання команд до історії без їхнього виконання.
\goodbreak

\subsection{\texttt{man}}
Вивід інструкції з використання програми. Загальна структура довідки \texttt{man} містить дев'ять розділів:
\begin{enumerate}
	\item Прикладні програми та консольні команди;
	\item Системні виклики ядра;
	\item Інформація про бібліотеки;
	\item Спеціальні файли, що знаходяться в директорії \texttt{/dev};
	\item Формати файлів (наприклад, \texttt{/etc/passwd});
	\item Ігри;
	\item Різне (макропакунки, домовленості, конвенції);
	\item Команди адміністрування, що виконуються користувачем \texttt{root};
	\item Ядро системи (розділ не є стандартизованим).
\end{enumerate}
Типове використання: \\
\texttt{man <команда>}
\goodbreak

\subsection{\texttt{echo}}
Команда, що виводить рядок тексту на стандартний вивід (зазвичай екран); часто використовується в скриптах для генерації сповіщень. Типове використання: \\
\texttt{echo "текст"} -- вивід тексту, що знаходиться в лапках.
\goodbreak

\subsection{\texttt{zip} та \texttt{unzip}}
Утилити для стискання файлів і створення zip-файлів (це є традиційний формат, що використовується скрізь); рівень компресії обирається автоматично.\\
\texttt{zip archive.zip file1 file2} -- створити файл archive.zip і запакувати в нього file1 і file2, що знаходяться в поточній директорії; \\
\texttt{unzip archive.zip} -- розпакувати archive.zip у поточну директорію.
\goodbreak

\subsection{\texttt{useradd}}
Утиліта з керування користувачами в системі. Для додавання нового користвувача використовуйте \\
\texttt{useradd -m -G <додаткові групи> <ім'я користувача>} \\
Опції команди: \\
\texttt{-m} або \texttt{--create-home} -- створити домашню директорію користувача (по замовчанню \texttt{/home/ім'я користувача}); \\
\texttt{-G} або \texttt{--groups} -- додати користувача до групи, наприклад, \\
\texttt{useradd -m -G docker,wheel user} -- створити користувача з іменем \texttt{user}, зробити йому домашню директорію \texttt{/home/user} і додати його в групи \texttt{docker} і \texttt{wheel} \\
\goodbreak

\subsection{\texttt{alias}}
Команда дозволяє створювати <<скорочення>> для команди, імені файлу або тексту. Під час виконання bash замінює це скорочення повною інструкцією, що було налаштовано. Використання: \\
\texttt{alias ім'я=значення}, наприклад \\
\texttt{alias up='sudo apt update \&\& sudo apt upgrade'} \\
Для скасування скорочення використовується команда \\
\texttt{unalias}, наприклад, \\
\texttt{unalias up}
\goodbreak

\subsection{\texttt{ps}}
Показує стан процесів, що наразі існують в системі; дані про процеси отримуються з віртуальних файлів файлової системи \texttt{/proc} \\
Запуск команди без опцій або аргументів надасть перелік процесів, що працюють, та інформацію про них: \\
\begin{itemize}
	\item Ідентифікаційний код процесу (Process ID або pid);
	\item Тип терміналу, в якому процес було запущено;
	\item Час роботи процесу;
	\item Команду, якою той процес було запущено.
\end{itemize}
\goodbreak


\section{Файлова структура}
Якщо ви раніше користувались лише Microsoft Windows, то вас може засмутити відсутність дисків, що позначені літерами, та не дуже зрозумілі з першого погляду імена директорій. \\ 
Типова структура директорій у будь-якому дистрибутиві Linux є певною мірою стандартизована. Розглянемо для прикладу дистрибутив Ubuntu Server 20.04:\\
\begin{verbatim}
$ ls -la /
total 8388692
drwxr-xr-x   19 root root       4096 Dec  2 22:37 .
drwxr-xr-x   19 root root       4096 Dec  2 22:37 ..
lrwxrwxrwx    1 root root          7 Aug 31 06:52 bin -> usr/bin
drwxr-xr-x    5 root root       4096 Dec  7 04:51 boot
drwxr-xr-x   19 root root       4380 Dec  7 05:53 dev
drwxr-xr-x  166 root root      12288 Dec  9 06:15 etc
drwxr-xr-x   55 root root       4096 Dec  7 06:00 home
lrwxrwxrwx    1 root root          7 Aug 31 06:52 lib -> usr/lib
lrwxrwxrwx    1 root root          9 Aug 31 06:52 lib32 -> usr/lib32
lrwxrwxrwx    1 root root          9 Aug 31 06:52 lib64 -> usr/lib64
lrwxrwxrwx    1 root root         10 Aug 31 06:52 libx32 -> usr/libx32
drwx------    2 root root      16384 Dec  2 22:34 lost+found
drwxr-xr-x    2 root root       4096 Aug 31 06:52 media
drwxr-xr-x    2 root root       4096 Aug 31 06:52 mnt
drwxr-xr-x    3 root root       4096 Dec  7 04:41 opt
dr-xr-xr-x 1379 root root          0 Dec  7 05:52 proc
drwx------    5 root root       4096 Dec  7 06:10 root
drwxr-xr-x   40 root root       1280 Dec  9 13:32 run
lrwxrwxrwx    1 root root          8 Aug 31 06:52 sbin -> usr/sbin
drwxr-xr-x    6 root root       4096 Aug 31 06:54 snap
drwxr-xr-x    2 root root       4096 Aug 31 06:52 srv
dr-xr-xr-x   13 root root          0 Dec  7 05:52 sys
drwxrwxrwt   14 root root       4096 Dec  9 13:32 tmp
drwxr-xr-x   14 root root       4096 Aug 31 06:53 usr
drwxr-xr-x   14 root root       4096 Dec  7 04:37 var
\end{verbatim}
Отже, розглянемо цю структуру більш детально, це допоможе вам впевнено орієнтуватись у файловій системі та знати, що і де знаходиться. Про перші стовпчики цього переліку буде окремий розділ \ref{Безпека та дозволи}, тому поки що сконцентруємось на останньому. Це всі директорії (та інколи файли), що є в кореневій директорії (вона позначається \texttt{/}) будь-якої UNIX-подібної системи. Назва <<коренева>> натякає на деревоподібність структури файлової системи, то ж ходімо гілками.

	\subsection{\texttt{.} та \texttt{..}}
	Директорії, що позначені крапкою та двома крапками, є, відповідно, поточною директорією та директорією, що знаходиться рівнем вище (але оскільки немає нічого рівнем вище за \texttt{/}, то тут ми маємо єдиний випадок, коли обидва ці посилання вказують на одну й ту саму директорію).
	\goodbreak
	
	\subsection{\texttt{/bin} та \texttt{/usr/bin}}
	Директорії з файлами, що виконуються.\\ Від самого початку, коли комп'ютери мали дуже обмежені ресурси, між цими директоріями була досить суттєва різниця, але вона зникла за останні двадцять років. В директорії \texttt{/bin} знаходились бінарні файли, наявність яких була обов'язковою для гарантованої роботи в режимі з одним користувачем (суперкористувач або, він же, root) в системі, тобто в режимі аварійному, до якого звертаються під час, коли щось пішло не так. В директорії \texttt{/usr/bin} знаходяться так само бінарні файли, але вже ті, що доступні для виконання звичайними користувачами. Наразі їхній вміст є загальним, що можна побачити наочно:\\
	\texttt{bin -> usr/bin}\\ означає, що одна директорія є посиланням на іншу.
	\goodbreak
	
	\subsection{\texttt{/boot}}
	Директорія, що містить статичні (тобто незмінні) файли, необхідні для завантаження системи. \\ Файли, що стосуються менеджера завантаження (наприклад, GRUB) і ядра, що встановлені, зберігаються тут. Варто пам'ятати, що конфігураційні файли, які відповідають менеджеру завантаження, знаходяться не тут, а в директорії \texttt{/etc} разом з іншими файлами налаштувань.
	\goodbreak
	
	\subsection{\texttt{/dev}}
	Файли пристроїв. \\
	Linux працює з пристроями як з файлами (взагалі в UNIX-подібних системах <<все є файлом>>, отже директорія \texttt{/dev} містить купу спеціальних файлів, що репрезентують пристрої. Вони не є файлами як такими, лише вдаються файлами: наприклад, перший SATA-диск в системі буде відображено як \texttt{/dev/sda}, якщо ви захочете його розмітити, то вам треба буде запустити відповідний редактор (наприклад, fdisk) і вказати \texttt{/dev/sda} як ціль для редагування. \\
	Крім звичайних пристроїв ця директорія містить псевдо-пристрої, які є віртуальними і не відповідають апаратному забезпеченню. Наприклад, \texttt{/dev/random} генерує випадкові числа, а \texttt{/dev/null} взагалі нічого не генерує, а навпаки, знищує все, що туди потрапляє, то ж якщо ви скеруєте куди вивід будь-якої команди, то його буде видалено.
	\goodbreak
	
	\subsection{\texttt{/etc}}
	Файли налаштувань. \\
	Файли, що зберігаються в цієї директорії, використовуються для налаштувань усього програмного та апаратного забезпечення в системі і можуть бути відредаговані з використанням звичайного текстового редактора. Майте на увазі, що конфігураційні файли в директорії \texttt{/etc} є глобальними (тобто діють на всю систему); конфігураційні файли для налаштувань конкретного користувача знаходяться в домашній директорії кожного користувача.
	\goodbreak
	
	\subsection{\texttt{/home}}
	Домашні директорії. \\
	Оскільки всі UNIX-подібні операційні системи із самого початку були розраховані на багатьох користувачів, що можуть працювати водночас, тому було застосовано низку заходів для відокремлення особистих просторів користувачів. Зазвичай кожен користувач, що може працювати в системі, має свою домашню директорію, що розташовано в \texttt{/home/<username>}. Це єдина директорія, де користувач може робити все; для того, щоб мати можливість редагувати файли за межами своєї домашньої директорії, користувач має стати суперкористувачем (\texttt{root}).
	\goodbreak
	
	\subsection{\texttt{/lib}, \texttt{/lib32}, \texttt{/lib64}, \texttt{/libx32} та \texttt{/usr/lib}, \texttt{/usr/lib32}, \texttt{/usr/lib64, \texttt{/usr/libx32}}}
	В сучасних дистрибутивах перша половина з перелічених директорій найчастіше є посиланням на другу половину (ситуація приблизно така сама, як з \texttt{/bin}), в цих директоріях зберігаються файли бібліотек, що є необхідними для роботи бінарних файлів, які розташовані в \texttt{/bin}, \texttt{/usr/bin}, \texttt{/sbin} та \texttt{/usr/sbin}.
	\goodbreak
	
	\subsection{\texttt{/lost+found}} -- файли, що було відновлено. \\
	Кожна файлова система в Linux має таку директорію. Якщо з файловою системою раптом станеться якійсь збій, то під час наступного завантаження буде виконано перевірку. Всі файли, що було пошкоджено, копіюються в директорію \texttt{lost+found} для того, щоб ви мали можливість відновити їх.
	\goodbreak
	
	\subsection{\texttt{/media}}
	Змінні пристрої. \\
	Директорія \texttt{/media} містить піддиректорії, куди монтуються змінні пристрої (USB флеш-накопичувачи, CD, DVD тощо). Якщо у вас в системі налаштовано автомонтування (а воно налаштовано по замовчанню майже скрізь), то будь-який накопичувач, щойно буде підключений, буде змонтовано в піддиректорію всередині \texttt{/mount} і ви матимете доступ до його вмісту саме там.
	\goodbreak
	
	\subsection{\texttt{/mnt}}
	Тимчасова точка монтування. \\
	Історично саме ця директорія призначалась для монтування системним адміністратором тимчасових файлових систем на час роботи з ними, але зараз директорію \texttt{/mnt} рекомендовано використовувати хіба що під час інсталяції деяких <<простих>> дистрибутивів, як от Gentoo або Arch, бо в решті випадків ви можете монтувати файлові системи будь-куди на ваш вибір.
	\goodbreak
	
	\subsection{\texttt{/opt}}
	Додаткові пакунки. \\
	Директорія \texttt{/opt} згідно з конвенціями використовується для проприєтарного програмного забезпечення, що встановлюється окремо і яке не дотримується стандартної ієрархії файлових систем.
	\goodbreak
	
	\subsection{\texttt{/proc}}
	Файли процесів та стану ядра. \\
	Директорія \texttt{/proc} схожа на \texttt{/dev} тим, що зберігає не звичайні файли, а спеціальні, що містять інформацію про процеси та стан системи, зокрема ядра.
	\goodbreak
	
	\subsection{\texttt{/root}}
	Домашня директорія суперкористувача. \\
	Наявність відокремленої домашньої директорії для суперкористувача має такі ж історичні підстави, як \texttt{/bin} або \texttt{/sbin} -- ця директорія має бути доступною в будь-якому випадку, якщо система працює в аварійному режимі. Якщо суперкористувач введе команду \texttt{cd $\sim$/}, то потрапить саме сюди.
	\goodbreak
	
	\subsection{\texttt{/run}}
	Файли стану застосунків. \\
	Директорія \texttt{/run} є досить новою, вона з'явилась в Linux десь років зо десять тому, наразі це стандартне місце для зберігання тимчасових файлів (наприклад, сокети або ідентифікатор процесу), які не можуть бути в \texttt{/tmp} тому що там їх можна вилучити (зокрема випадково).
	\goodbreak
	
	\subsection{\texttt{/sbin}}
	Бінарні файли для системного адміністрування. \\
	Директорія \texttt{/sbin} схожа на \texttt{/bin}; в неї зберігаються деякі бінарні файли, про можуть бути запущені суперкористувачем для системного адміністрування.
	\goodbreak
	
	\subsection{\texttt{/srv}}
	Дані сервісів. \\
	В директорії \texttt{/srv} по замовчанню знаходяться дані сервісів, що забезпечує система. Наприклад, директорія з даними для вебсайту -- \texttt{/srv/http}.
	\goodbreak
	
	\subsection{\texttt{/tmp}}
	Тимчасові файли. \\
	Програми зберігають свої тимчасові файли в директорії \texttt{/tmp}. Ці файли переважно вилучаються під час завантаження операційної системи та можуть бути вилучені в будь-який час уручну або утилітами на кшталт \texttt{tmpwatch}.
	\goodbreak
	
	\subsection{\texttt{/usr}}
	Бінарні файли користувачів та дані лише для читання. \\
	Частково про вміст цієї директорії було написано в розділі <<\texttt{/bin} та \texttt{/usr/bin}>>, але завжди є, що додати. Директорія \texttt{/usr} містить, наприклад, файли, що не залежать від архітектури (вони зберігаються в \texttt{/usr/share}) та файли, які є локально скомпільованими додатками, що встановлені в системі (\texttt{/usr/local}) -- це запобігає засмічуванню системи.
	\goodbreak
	
	\subsection{\texttt{/var}}
	Файли даних, що змінюються. \\
	На відміну від директорії \texttt{/usr}, яка є лише для читання, \texttt{/var} є директорією, куди різні процеси або програми можуть писати свої дані, наприклад, в піддиректорії \texttt{/var/log} зберігаються файли логів.
	\goodbreak
	
\section{Безпека та дозволи} \label{Безпека та дозволи}

\section{Керування процесами та пристроями}
\section{Файлові системи та керування ними}
\section{Додаткові відомості}
\tableofcontents
\doclicenseThis
\end{document}